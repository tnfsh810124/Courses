\documentclass[10pt]{article}
\usepackage[usenames]{color} %used for font color
\usepackage{amssymb} %maths
\usepackage{amsmath} %maths
\usepackage[utf8]{inputenc} %useful to type directly diacritic characters
\begin{document}
Showing that the Lasso estimate for $\beta$ is the mode under this posterior distribution is the same thing as showing that the most likely value for $\beta$ is given by the lasso solution with a certain $\lambda$. We can do this by taking our likelihood and posterior and showing that it can be reduced to the canonical Lasso Equation from the book.
Let's start by simplifying it by taking the logarithm of both sides:
\begin{align*}
    &\log
    f(Y \mid X, \beta)p(\beta)\\
    =&
    \log
    \left[
        \left(
            \frac{
                1
            }{
                \sigma \sqrt{2\pi}
            }
        \right)^n
        \left(
            \frac{
                1
            }{
                2b
            }
        \right)
        \exp
        \left(
            - \frac{
                1
            }{
                2\sigma^2
            }
            \sum_{i = 1}^{n}
            \left[
                Y_i - (\beta_0 + \sum_{j = 1}^{p} \beta_j X_{ij})
            \right]^2
            -
            \frac{
                \lvert \beta \rvert
            }{
                b
                }
        \right)
    \right]
    \\
    =&
    \log
    \left[
        \left(
            \frac{
                1
            }{
                \sigma \sqrt{2\pi}
            }
        \right)^n
        \left(
            \frac{
                1
            }{
                2b
            }
        \right)
    \right]
    -
    \left(
        \frac{
            1
        }{
            2\sigma^2
        }
        \sum_{i = 1}^{n}
        \left[
            Y_i - (\beta_0 + \sum_{j = 1}^{p} \beta_j X_{ij})
        \right]^2
        +
        \frac{
            \lvert \beta \rvert
        }{
            b
        }
    \right)
\end{align*}

\end{document}