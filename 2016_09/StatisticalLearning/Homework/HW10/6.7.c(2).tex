\documentclass[10pt]{article}
\usepackage[usenames]{color} %used for font color
\usepackage{amssymb} %maths
\usepackage{amsmath} %maths
\usepackage[utf8]{inputenc} %useful to type directly diacritic characters
\begin{document}
\qquad We want to maximize the posterior, this means:
\begin{align*}
    &\arg\max_\beta \, f(\beta \mid X, Y)\\
    \propto &
    \arg\max_\beta
    \,
    \log
    \left[
        \left(
            \frac{
                1
            }{
                \sigma \sqrt{2\pi}
            }
        \right)^n
        \left(
            \frac{
                1
            }{
                2b
            }
        \right)
    \right]
    -
    \left(
        \frac{
            1
        }{
            2\sigma^2
        }
        \sum_{i = 1}^{n}
        \left[
            Y_i - (\beta_0 + \sum_{j = 1}^{p} \beta_j X_{ij})
        \right]^2
        +
        \frac{
            \lvert \beta \rvert
        }{
            b
            }
    \right)
    \\
\end{align*}
\qquad Since we are taking the difference of two values, the maximum of this value is the equivalent to taking the difference of the second value in terms of $\beta$. This results in:
\begin{align*}
    &=
    \arg\min_\beta
    \,
    \frac{
        1
    }{
        2\sigma^2
    }
    \sum_{i = 1}^{n}
    \left[
        Y_i - (\beta_0 + \sum_{j = 1}^{p} \beta_j X_{ij})
    \right]^2
    +
    \frac{
        \lvert \beta \rvert
    }{
        b
    }
    \\
    &=
    \arg\min_\beta
    \,
    \frac{
        1
    }{
        2\sigma^2
    }
    \sum_{i = 1}^{n}
    \left[
        Y_i - (\beta_0 + \sum_{j = 1}^{p} \beta_j X_{ij})
    \right]^2
    +
    \frac{
        1
    }{
        b
    }
    \sum_{j = 1}^{p} \lvert \beta_j \rvert
    \\
    &=
    \arg\min_\beta
    \,
    \frac{
        1
    }{
        2\sigma^2
    }
    \left(
        \sum_{i = 1}^{n}
        \left[
            Y_i - (\beta_0 + \sum_{j = 1}^{p} \beta_j X_{ij})
        \right]^2
        +
        \frac{
            2\sigma^2
        }{
            b
        }
        \sum_{j = 1}^{p} \lvert \beta_j \rvert
    \right)
\end{align*}
\qquad By letting $\lambda = 2\sigma^2/b$, we can see that we end up with:
\begin{align*}
    &=
    \arg\min_\beta
    \,
    \sum_{i = 1}^{n}
    \left[
        Y_i - (\beta_0 + \sum_{j = 1}^{p} \beta_j X_{ij})
    \right]^2
    +
    \lambda
    \sum_{j = 1}^{p} \lvert \beta_j \rvert
    \\
    &=
    \arg\min_\beta
    \,
    \text{RSS}
    +
    \lambda
    \sum_{j = 1}^{p} \lvert \beta_j \rvert
\end{align*}
\qquad That is what we know is the Lasso from Equation 6.7 in the book. Thus we know that when the posterior comes from a Laplace distribution with mean zero and common scale parameter $b$, the mode for $\beta$ is given by the Lasso solution when $\lambda = 2\sigma^2 / b$. 
\end{document}